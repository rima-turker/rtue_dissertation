\cleardoublepage
\section*{}
\addcontentsline{toc}{section}{Belehrung}
\thispagestyle{plain}
\vspace{-3cm}
\begin{flushright}
  \noindent
  Lei Zhang\\
  Gr\"otzinger Str. 19\\
  76227 Karlsruhe\\
\end{flushright}

\vspace{0.5cm}

\begin{center}
\textbf{\Large{Eidesstattliche Versicherung}}\\
\vspace{0.5cm}
\textbf{\Large{Belehrung}}
\vspace{0.5cm}

\noindent
gem\"a\ss ~\S 6 Abs. 1 Ziff. 5 der Promotionsordnung des Karlsruher
Instituts f\"ur Technologie f\"ur die Fakult\"at f\"ur
Wirtschaftswissenschaften

\end{center}

%\vspace{0.5cm}

\noindent
Die Universit\"aten in Baden-W\"urttemberg verlangen eine Eidesstattliche
Versicherung \"uber die Eigenst\"andigkeit der erbrachten
wissenschaftlichen Leistungen, um sich glaubhaft zu versichern, dass
der Promovend die wissenschaftlichen Leistungen eigenst\"andig erbracht
hat.

\vspace{0.2cm}
\noindent
Weil der Gesetzgeber der Eidesstattlichen Versicherung eine besondere
Bedeutung bei\-misst und sie erhebliche Folgen haben kann, hat der
Gesetzgeber die Abgabe einer falschen eidesstattlichen Versicherung
unter Strafe gestellt. Bei vors\"atzlicher (also wissentlicher) Abgabe
einer falschen Erkl\"arung droht eine Freiheitsstrafe bis zu drei Jahren
oder eine Geldstrafe.

\vspace{0.2cm}
\noindent
Eine fahrl\"assige Abgabe (also Abgabe, obwohl Sie h\"atten erkennen
m\"ussen, dass die Erkl\"arung nicht den Tatsachen entspricht) kann eine
Freiheitsstrafe bis zu einem Jahr oder eine Geldstrafe nach sich
ziehen.

\vspace{0.2cm}
\noindent
Die entsprechenden Strafvorschriften sind in \S 156 StGB (falsche
Versicherung an Eides Statt) und in \S 161 StGB (fahrl\"assiger
Falscheid, fahrl\"assige falsche Versicherung an Eides Statt)
wiedergegeben.

\vspace{0.2cm}
\noindent
\S 156 StGB: Falsche Versicherung an Eides Statt \\
Wer vor einer zur Abnahme einer Versicherung an Eides Statt
zust\"andigen Beh\"orde eine solche Versicherung falsch abgibt oder unter
Berufung auf eine solche Versicherung falsch aussagt, wird mit
Freiheitsstrafe bis zu drei Jahren oder mit Geldstrafe bestraft.

\vspace{0.2cm}
\noindent
\S 161 StGB: Fahrl\"assiger Falscheid, fahrl\"assige falsche Versicherung
an Eides Statt \\
Abs.~1: Wenn eine der in den \S154 bis 156 bezeichneten Handlungen aus
Fahrl\"assigkeit begangen worden ist, so tritt
Freiheitsstrafe bis zu einem Jahr oder Geldstrafe ein. \\
Abs.~2: Straflosigkeit tritt ein, wenn der T\"ater die falsche Angabe
rechtzeitig berichtigt. Die Vorschriften des \S158 Abs. 2 und 3 gelten
entsprechend.

\vspace{1cm}

\noindent Ort und Datum \hspace{5cm} Unterschrift


