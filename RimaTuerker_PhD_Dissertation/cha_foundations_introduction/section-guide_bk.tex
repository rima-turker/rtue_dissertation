\section{Guide to the Reader}

This thesis comprises ten chapters, which are divided into six parts according to the addressed tasks. Besides Chapter 1 and Chapter 2, which provide the foundations of this thesis, each of the following core chapters (Chapter 3 - Chapter 9) covers one of the research questions stated before and Chapter 10 provides the conclusions.

%This thesis comprises ten chapters, which are divided into six parts according to the addressed tasks. Besides Chapter 2 and Chapter 3, Besides Chapter 2 and Chapter 3, which provide the foundations of this thesis, the following chapters are self-contained and cover one of the research question stated before.

\paragraph{\textbf{Part~\ref{part:foundations}}} provides the \textbf{Foundations} for this thesis. 

\begin{itemize}
\item \textbf{Chapter \ref{cha:introduction}.} We introduce the challenges and tasks concerned in this thesis, break down the principal research question into seven individual research questions, present the main contributions, and provide this guide to the reader.
\item \textbf{Chapter \ref{cha:foundations_basics}.} We provide a brief introduction to knowledge bases and preliminaries for the tasks of semantic annotation, semantic search and semantic-based IR. 
\end{itemize}

\paragraph{\textbf{Part~\ref{part:annotation}}} discusses the task of \textbf{Semantic Annotation}.

\begin{itemize}
\item \textbf{Chapter \ref{cha:anno_disambiguation}.} We show that our approach to entity disambiguation achieves promising results 
by leveraging the contextual entities derived from the given document and collective algorithms based on Markov chains.
\item \textbf{Chapter \ref{cha:anno_salient}.} We present a graph-based disambiguation framework of salient entity discovery and linking by utilizing a topic-sensitive model based on Wikipedia categories.
\end{itemize}

\paragraph{\textbf{Part~\ref{part:search}}} discusses the task of \textbf{Semantic Search}.

\begin{itemize}
\item \textbf{Chapter \ref{cha:search_entity}.} We present the first probabilistic model that takes both relevance and timeliness into consideration for entity recommendation given temporal information needs.
\item \textbf{Chapter \ref{cha:search_relational}.} We show that query rewriting can help to improve not only the result quality but also the runtime performance of keyword search on knowledge graphs.
\end{itemize}

\paragraph{\textbf{Part~\ref{part:cross-lingual}}} discusses the \textbf{Cross-lingual} extensions of \textbf{Semantic Annotation and Search} as well as the task of \textbf{Semantic-based Cross-lingual IR}.

\begin{itemize}
\item \textbf{Chapter \ref{cha:anno_framework}.} We demonstrate an infrastructure of cross-lingual entity linking, which supports both service-oriented and user-oriented interfaces for annotating Web documents in different languages with entities from Wikipedia and Linked Open Data (LOD).
\item \textbf{Chapter \ref{cha:search_interpretation}.} We present a knowledge base approach to cross-lingual query interpretation by transforming query keywords in different languages to their semantic representation.
\end{itemize}

%\paragraph{\textbf{Part~\ref{part:application}}} discusses the application of \textbf{Semantic-based Cross-lingual IR}. 
%especially in the \textbf{Cross-lingual} setting. 

\begin{itemize}
%\item \textbf{Chapter \ref{cha:anno_framework}.} 
\item \textbf{Chapter \ref{cha:ir_framework}.} We demonstrate a framework of cross-lingual IR by exploiting entities in multilingual knowledge bases, which serve as an interlingua to connect keyword queries and Web documents across languages.
\end{itemize}

\paragraph{\textbf{Part~\ref{part:conclusions}}} concludes this thesis.

\begin{itemize}
\item \textbf{Chapter \ref{cha:conclusions_summary}.} The thesis ends with a summary of the main conclusions and an outlook on future research directions.
\end{itemize}

%\paragraph{\textbf{Part~\ref{part:appendix}}} contains additional material in an \textbf{Appendix}.