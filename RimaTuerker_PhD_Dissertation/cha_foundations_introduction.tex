\chapter{Introduction} \label{cha:introduction}
The World Wide Web also known as Web has become one of the largest global collection of information %interconncete knowledge 
with over 1.5 billion websites\footnote{https://www.internetlivestats.com/total-number-of-websites}. The Web is a global network environment where documents and other Web sources are identified by Uniform Resource Locators (URLs)~\cite{berners1998uri}. The documents are interlinked by hyper links~\cite{jacobs2004architecture} and accessible over the Internet. This global document repository encompasses almost every topic of human interest. The usefulness of Web document access can be seen from the fact that Google processes over 40,000 search queries every second on average, which translates to over 3.5 billion searches per day and 1.2 trillion searches per year worldwide\footnote{https://www.internetlivestats.com/google-search-statistics/}. 

Furthermore, today, billions  of  users$^1$ access and even contribute to this massive information exchange platform. Due to the large number of contributions as well as the digitization of all areas the content of the Web is drastically multiplying~\cite{STCImprovedby}. %Thus, there arouses a necessity of automatic analyzing and processing techniques of such data in order to satisfy a particular need of information~\cite{STTopicMemory}.
As a result, the fields of Natural  Language  Processing (NLP)~\cite{Jurafsky:2009:SLP:1214993} evolves in parallel, which concern the automatic analyzing and processing of natural language documents in order to satisfy a particular need of information. In other words, NLP is a field of computer science, artificial intelligence, and computational linguistics whose techniques are specifically designed to extract meaningful information from natural language documents. Some examples of NLP applications are Spelling and Grammar Checking, Auto  completion, Text Categorization, Text Summarization, Information Retrieval, etc.   Among those applications, text classification is the fundamental one which has been proven to be useful in various applications, such as sentiment analysis, news feed filtering and categorization, etc.~\cite{STTopicMemory}.

%Importance necessity of classification Application
Text categorization (a.k.a text classification) aims to assign one or more predefined classes or categories to natural language documents based on the attributes of the each text document. The textual data can be anything ranging from a phrase, sentences, paragraphs or an entire document. %Due to recent advancement of NLP many researchers are now concerned with developing new applications by exploiting text classification methods Some examples of text classification... 
However, with the rapid growth of online platforms such as Facebook, Twitter, online forums, etc. the Web users are now generating more and more \textit{short text} data. A large body of daily generated content of the Web, such as short text messages,micro blog posts, search queries etc. have become an important form for individuals to share information~\cite{STTopicMemory}. The main characteristic of short text is that the length of the text is very limited which is no longer than 200 characters~\cite{song2014short}. Further, unlike other documents short texts are usually rather ambiguous and sparse. Due to the limited length of the text which is only several to a dozen words~\cite{song2014short} the ambiguity cannot be resolved by depending on the context. Further, since the context is rather limited feature extraction is rather difficult. Such characteristics pose several major challenges to short text categorization tasks. While conventional text classification methods such as Support Vector Machines (SVMs) have demonstrated their success in classifying long and well structured text, as e.g., news articles, in case of short text they seem to have a substandard performance~\cite{STTopicMemory}.


\par Recently, several deep learning approaches have been proposed for short text classification, which demonstrated remarkable performance in this task~\cite{sentiment/aaai/MaPC18,DBLP:conf/emnlp/ChenSBY17}. The two main advantages of these models for the classification task are that minimum effort is required for feature engineering and their classification performance is better in comparison to traditional text classification approaches~\cite{WeaklySupervised}. However, the requirement of large amounts of labeled data remains the main bottleneck for neural network based approaches~\cite{WeaklySupervised}. Acquiring labeled data for the classification task is costly and time-consuming. Especially, if the data to be labeled is of a specific domain then only a limited number of domain experts is able to label them correctly, which makes it a labor intensive task.

In this thesis, we are concerned with
 %For instance, Google processes over 40,000 search queries every second on average which translates to over 3.5 billion searches per day. Maybe also news feed info

\section{Challenges and Tasks} \label{sec:challenges}

\noindent  \textbf{Challenge 1: Requirement of Labeled Training Data.} 
Over the last few years, with the advent of the deep learning techniques, industry,
government, and academia exceedingly have been focused on developing machine leaning based systems. %In fact the Worldwide estimated spend in 2019 on AI based systems \$35.8 billion\footnote{https://www.idc.com/getdoc.jsp?containerId=prUS44911419}. 
Furthermore, recently proposed supervised text classification approaches especially deep nn*ref approaches have demonstrated extremely superior performance in this task. Despite the success of these models they can easily consume million-scale labeled data~\cite{WeaklySupervised}. In other words,
Many classification models suffer from absence of labeled data. Despite the popularity of neural networks  However, due to high cost of human labelling it is hard to obtain training data~\cite{transfer_learning_for_text_Classification}. labelling trainign daa is mot costly snorkel especially when the domain experts are required
In order to overcome the bottleneck of labeled data

\noindent  \textbf{Task 1: Utilizing Knowledge Bases as an External Source.} 
Traditional supervised methods require a significant amount of training data and manually labeling such data can be very time-consuming and costly. To overcome the requirement for labeled data, Knowledge Bases (KBs) can be leveraged to perform short text categorization without using any labeled data. In other words, the semantic relation between the  entities  represented  in  a  short  text  and  the  predefined categories can be utilized to drive the category of the text. Such semantic similarity can be quantified with a help of KB such as Wikipedia which contains entities that are associated with hierarchically related categories.

%\noindent  \textbf{Task 1.2: Utilizing the Semantic Similarity Between the Text and Predefined Labels to Perform the Categorization.} 

\noindent  \textbf{Challenge 2: Limited Context and non-standard characteristics.%\noindent  \textbf{Challenge 2: Learning supervised models without requiring any labeled data as a prerequisite. 
%under weak supervision without requiring any labeled data as a prerequisite.
} With the extensive growth of online platforms people are generating everyday more short text data such as product comments, news title, short text messages, tweets, etc. To be able to make sense of such data many researchers have been focusing on categorizing of short texts~\cite{...}. Usually, the content of the short texts contain rather non-standard terms and noise~\cite{song2014short}. Further, the main characteristics of the short text i.e., limited context, sparsity and ambiguity pose much more challenges to the categorization task. 
%The traditional text categorization models~\cite{...} mostly focused on the categorization of the arbitrary length of documents. 
Thus, to overcome those challenges,it  is  indispensable  to  use  external  sources  such  as  Knowledge  Bases  (KBs)  to enrich and obtain more advanced text representations~\cite{deepShort}.

\noindent  \textbf{Task 2: Enriching text representations by utilizing KBs.} 
The traditional text classification models represent text as a bag-of-words to perform text classification. However in the case of short text where the context is rather limited and the ambiguity is one of the major problems, such approaches that utilize only words often lead to inaccurate results. 
Further, those approaches do not consider the semantic relation between the words. However, entities carry much more information than the words. Especially, when we consider the link structure of KBs the relation between the entities, etc. can represent the text better. Especially, when we consider the structure of the KBs such as entity relations, entities category associations can be extremely helpful to enrich the semantic representation of the short texts.    
%\noindent  \textbf{Task 2.1: Combining different resources to generate labeled training data.} 


%\noindent  \textbf{Task 2.2: Adaption of neural models with pseudo-labeled training data} 

%\noindent  \textbf{Challenge 3: Hierarchically related a large number of labels (patent).} 


%\noindent  \textbf{Task 3.1: -} 

\section{Research Questions} \label{sec:questions}

The principal research question of this thesis is:
\begin{restatable}{research*}{rstqprincipal} \label{q:qprincipal}
\vspace{1em}~\\
\forceindent \textbf{\emph{How to perform short text categorization without requiring any hand-labeled data?}}
\vspace{1em}
\end{restatable}

This broad research question is broken down into four specific research questions, each of which entails an combination of challenges and tasks as stated above and will be addressed in the remainder of this thesis.

\noindent The first two research questions are derived from Challenge 1 \emph{Requirement of Labeled Training Data} and concerns Task 1 \emph{Utilizing Knowledge Bases as an External Source}:

\begin{restatable}{research}{rstqdisambiguation} \label{q:disambiguation}
How can a KB be utilized for short text categorization without requiring any labeled data?
\end{restatable}
\vspace{-0.9em}
\begin{restatable}{research}{rstqsalient} \label{q:salient}
How to capture the semantic relation between text and predefined labels?
\end{restatable}
\vspace{-0.9em}
----


The next two research questions are derived from Challenge 1 \emph{Requirement of Labeled Training Data} and Challenge 2 \emph{Limited Context and non-standard characteristics} and concerns Task 1 \emph{Utilizing Knowledge Bases as an External Source} and Task 2 \emph{Enriching text representations by utilizing KBs}:

\begin{restatable}{research}{rstqrecommendation} \label{q:recommendation}
How can KBs be exploited to create labeled data for supervised methods?
\end{restatable}
\vspace{-0.9em}
\begin{restatable}{research}{rstqrecommendation} \label{q:recommendation}
How to enrich a short text representation by leveraging a KB as an external source?
\end{restatable}
\vspace{-0.9em}



\section{Contributions of the Thesis}

\begin{restatable}{contribution}{Cdisambiguation} \label{c:disambiguation}
A new paradigm for short text categorization, based on a knowledge base
\end{restatable}
\vspace{-0.9em}
\noindent 

\begin{restatable}{contribution}{Csalient} \label{c:salient}
A probabilistic model for short text categorization
\end{restatable}
\vspace{-0.9em}

%Combining different resources to generate labeled training data.
\begin{restatable}{contribution}{Crecommendation} \label{c:recommendation}
An approach to combine resources to generate labeled data which can be used for any arbitrary classification model
\end{restatable}
\vspace{-0.9em}

\begin{restatable}{contribution}{Csearch} \label{c:search}
A method which utilizes knowledge base sources to enhance the semantic representation of short texts  
\end{restatable}
\vspace{-0.9em}


\section{Publications} \label{sec:publications}
\section{Guide to the Reader}

-

\paragraph{\textbf{Part~\ref{part:foundations}}} - 

\begin{itemize}
\item \textbf{Chapter~\ref{cha:introduction}.} -
\item \textbf{Chapter~\ref{cha:foundations_basics}.}
\end{itemize}

\paragraph{\textbf{Part~\ref{part:annotation}}} -



\paragraph{\textbf{Part~\ref{part:search}}} -.

\paragraph{\textbf{Part~\ref{part:cross-lingual}}} -.

\paragraph{\textbf{Part~\ref{part:conclusions}}} concludes this thesis.

\begin{itemize}
\item \textbf{Chapter~\ref{cha:conclusions_conclusions}.} The thesis ends with a summary of the main conclusions and an outlook on future research directions.
\end{itemize}

%\paragraph{\textbf{Part~\ref{part:appendix}}} contains additional material in an \textbf{Appendix}.